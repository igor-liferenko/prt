\pdfhorigin=1cm
\hsize=\pdfpagewidth \advance\hsize by -2\pdfhorigin
\dimen0=\pdfvorigin \pdfvorigin=1cm \advance\dimen0 by-\pdfhorigin
\advance\vsize by 2\dimen0

\catcode`\"=13 \def"{\leavevmode\hbox\bgroup\let"=\egroup\def\do##1{\catcode`##1=12}\dospecials\tt}
\nopagenumbers

\newcount\lineno
\def\setupverbatim{\tt
  \obeylines
  \def\do##1{\catcode`##1=12 } \dospecials
  \obeyspaces}
{\obeyspaces\global\let =\ }
\def\listing#1{\par\begingroup\setupverbatim\input#1 \endgroup}
{
  \everypar={\advance\lineno by 1 \ifnum\lineno<60 \llap{\sevenrm \the\lineno\quad}\fi}
  \listing{stest.h}
}
\medskip
\noindent{\bf Overview:}\par
\item{$\bullet$} Lines 36--37 is the way to tell if "#1"'s catcode is 13.
We use "\uppercase" to turn "#1" into a macro ("~" is a macro by default).
The matter is that in "\ifcat" the meaning of each token is used,
not the token itself. The exception is a macro: if the token is a macro, then the token
itself is used (and to prevent it from being expanded according to the rules of "\ifcat",
the token must be preceded by "\noexpand").

\item{$\bullet$} Undelimited explicit spaces are ignored in macro arguments, so in parameter text for "\sss"
we use explicit space as delimiter. The effect is the following: if first token in token list is explicit
space, it is matched to the space in parameter text and hence "#1" will be empty. On the other
hand,
if the first token
in token list is implicit space, an explicit space in the token list (if there is one --- not in the first position) or the space after the "!" (if there is no explicit space in token list) is matched to the space in parameter text, and "#1" will include all tokens in the stream before the explicit space.

\item{$\bullet$} Each space token (if it is the first in token list) is handled in two stages.
In the first stage (via "\futurelet") we detect catcode 10 in line 10.
In the second stage (via argument "#1" to macros "\sss" and "\ssss") we detect the other properties.

\item{$\bullet$} Let $\cal A$ be a character with catcode 13. For $\cal A$'s meaning there are two cases:
\itemitem{[1]} normal space
\itemitem{[2]} funny space

\item{} For $\cal A$'s charcode there are two cases:
\itemitem{[a]} charcode of $\cal A$ is 32 (this case may be split into two sub-cases: when
charcode of $\cal A$'s meaning is 32; and when not 32).
\itemitem{[b]} charcode of $\cal A$ is equal to charcode of $\cal A$'s meaning
(the charcode must not be 32 --- this is a sub-case of [a]; and it must not be 126 --- it is used in line 37).

\item{$\bullet$} Argument list for "\tact" is formed by
running "\string" on the first token in token list (it is assumed that "\escapechar" is between 0 and 255). If this token is not active, there will be two or more tokens in arument list. If this token is an active character with
charcode not 32, there will be one token in argument list; if this token is an active character with charcode 32, argument list will be empty because undelimited explicit spaces are ignored in macro arguments. "\tact" simply checks if "#2" is "\s" or if "#1" is "\s",
in order to determine whether the token is an active character.

\item{$\bullet$} Test in line 35 is passed by $\cal A$, for which charcode of $\cal A$ is the same as charcode of $\cal A$'s meaning,
and by explicit spaces (in line 34 care is taken to
guarantee that the meaning of a control sequence token is not "\escapechar",
because we do not want control sequence tokens to pass the test in line 35;
if "#1" is a control sequence token, extra tokens produced by "\string" are ignored,
because the test in line 35 is false in such case).

\item{$\bullet$} "\long" is used in case token list contains "\par".
\vfil\eject

\noindent\hrulefill\quad 0. Output is empty.\quad \hrulefill\null
\smallskip
\listing{stest0.tex}
\medskip
\noindent Test in line 10 is false.
\bigbreak
\noindent\hrulefill\quad 1. Output is {\tt SPACE EXPLICIT}\quad \hrulefill\null
\smallskip
\listing{stest1.tex}
\medskip
\noindent Test in line 10 is true. Test in line 18 is true. Test in line 26 is true.
\bigbreak
\noindent\hrulefill\quad 2. Output is {\tt SPACE FUNNY EXPLICIT}\quad \hrulefill\null
\smallskip
\listing{stest2.tex}
\medskip
\noindent Test in line 10 is true. Test in line 18 is false. Test in line 35 is true.
Test in line 37 is false.
\bigbreak
\noindent\hrulefill\quad 3. Output is {\tt SPACE}\quad \hrulefill\null
\smallskip
\listing{stest3.tex}
\medskip
\noindent Test in line 10 is true. Test in line 18 is true. Test in line 26 is false.
Test in line 52 is false.
Test in line 55 is false.
\bigbreak
\noindent\hrulefill\quad 4. Output is {\tt SPACE FUNNY}\quad \hrulefill\null
\smallskip
\listing{stest4.tex}
\medskip
\noindent Test in line 10 is true. Test in line 18 is false. Test in line 35 is false.
Test in line 52 is false.
Test in line 55 is false.
\bigbreak
\noindent\hrulefill\quad 5. Output is {\tt SPACE ACTIVE}\quad \hrulefill\null
\smallskip
\listing{stest5.tex}
\medskip
\noindent Test in line 10 is true. Test in line 18 is true. Test in line 26 is false.
Test in line 52 is false.
Test in line 55 is true.
\bigbreak
\noindent\hrulefill\quad 6. Output is {\tt SPACE ACTIVE}\quad \hrulefill\null
\smallskip
\listing{stest6.tex}
\medskip
\noindent Test in line 10 is true. Test in line 18 is true. Test in line 26 is false.
Test in line 52 is true.
\bigbreak
\noindent\hrulefill\quad 7. Output is {\tt SPACE FUNNY ACTIVE}\quad \hrulefill\null
\smallskip
\listing{stest7.tex}
\medskip
\noindent Test in line 10 is true. Test in line 18 is true. Test in line 35 is false.
Test in line 52 is false.
Test in line 55 is true.
\bigbreak
\noindent\hrulefill\quad 8. Output is {\tt SPACE FUNNY ACTIVE}\quad \hrulefill\null
\smallskip
\listing{stest8.tex}
\medskip
\noindent Test in line 10 is true. Test in line 18 is false. Test in line 35 is false.
Test in line 52 is true.
\bigbreak
\noindent\hrulefill\quad 9. Output is {\tt SPACE FUNNY ACTIVE}\quad \hrulefill\null
\smallskip
\listing{stest9.tex}
\medskip
\noindent Test in line 10 is true. Test in line 18 is false. Test in line 35 is true.
Test in line 37 is true.
\bye
