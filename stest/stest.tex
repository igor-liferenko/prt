\pdfhorigin=1cm
\hsize=\pdfpagewidth \advance\hsize by -2\pdfhorigin
\dimen0=\pdfvorigin \pdfvorigin=1cm \advance\dimen0 by-\pdfhorigin
\advance\vsize by 2\dimen0

\catcode`\"=13 \def"{\leavevmode\hbox\bgroup\let"=\egroup\def\do##1{\catcode`##1=12}\dospecials\tt}
\nopagenumbers

\newcount\lineno
\def\setupverbatim{\tt
  \obeylines
  \def\do##1{\catcode`##1=12 } \dospecials
  \obeyspaces}
{\obeyspaces\global\let =\ }
\def\listing#1{\par\begingroup\setupverbatim\input#1 \endgroup}
{
  \everypar={\advance\lineno by 1 \ifnum\lineno<60 \llap{\sevenrm \the\lineno\quad}\fi}
  \listing{stest.h}
}
\medskip
$$\vbox{\hsize=8.7cm
\it\noindent NOTE: implicit space can be either a control sequence (``implicit-cs'') or an active character (``implicit-active'')}$$
\medskip
\noindent{\bf Overview:}\par
\item{$\bullet$} Lines 36--37 is the way to tell if catcode of a character token is 13.
We do not do "\def~{}" before the "\ifcat" test because "~" is a macro by default. Without "\uppercase" implicit space character will have catcode 10. \footnote\dag{Note, that "\ifnum\catcode=13" can't be used to detect active character, because catcode is taken not from token, but from global charcode-catcode mapping (which may be different from
 the token's catcode when the check is performed).}
\item{$\bullet$} In "\testactive" we branch between implicit-cs and implicit-active.
\item{$\bullet$} Undelimited explicit spaces are ignored in macro arguments, so in parameter text for "\sss"
we use explicit space as delimiter. The effect is the following: if first token in token list is explicit
space, it is matched to the space in parameter text and hence "#1" will be empty. On the other
hand,
if the first token
in token list is implicit space, an explicit space in the token list (if there is one --- not in the first position) or the space after the "!" (if there is no explicit space in token list) is matched to the space in parameter text, and "#1" will include all tokens in the stream before the explicit space.
\item{$\bullet$} Each space token (if it is the first in token list) is handled via two separate routes.
In the first route
(via "\futurelet", which makes "\next" an implicit space token, regardless whether the space token is explicit or implicit) we detect catcode 10 in line 10 (and also normal or funny).
In the second route (via argument "#1" to macros "\sss" and "\ssss", which leaves each
space token intact) we detect
explicit or implicit-active.
\item{$\bullet$} For ``implicit-active'' there are two cases (let's assume that active character is `"+"'):
\itemitem{[a]} "+" $\rightarrow$ normal
\itemitem{[b]} "+" $\rightarrow$ funny
\item{} There are two more special cases:
\itemitem{[c]} charcode of active character is 32 (this case may be split into two sub-cases: when
the space, that is inside the active character, has charcode 32; and when charcode is not 32).
\itemitem{[d]} charcode of active character and charcode, which is inside of the active character, are the same (they must not be equal to 32 --- this is a sub-case of [c]; and they must not be equal to 126 --- it is used in line 37).
\item{$\bullet$} Argument list for "\tact" is formed by
running "\string" on the first token in token list (it is assumed that "\escapechar" is between 0 and 255). If this token is not active, there will be two or more tokens in arument list. If this token is an active character with
charcode not 32, there will be one token in argument list; if this token is an active character with charcode 32, argument list will be empty because undelimited explicit spaces are ignored in macro arguments. "\tact" simply checks if "#2" is "\s" or if "#1" is "\s",
in order to determine whether the token is an active character.
\item{$\bullet$} Explicit spaces are separated from implicit-cs via "\string"$+$"\if" (here care is taken to
guarantee that implicit-cs space does not have charcode of "\escapechar" because we do not want implicit-cs to pass this test; extra tokens produced by "\string" in implicit-cs case are ignored because the "\if"-test is false in such case). \footnote\ddag{Although the "\string"$+$"\if" test is not pure (some implicit-active spaces pass along with explicit ones), we use it first in the `funny' branch. If we tried to be more straightforward, we could
first separate implicit-cs from explicit+implicit-active like in "\testactive"$+$"\tact" via "\string", but then it would be impossible to tell implicit-active from explicit because "\string"
makes catcode 12 for both.} Explicit spaces are separated from implicit-active via
"\uppercase"$+$"\ifcat".
\item{$\bullet$} "\long" is used in case token list contains "\par".
\vfil\eject
\noindent\hrulefill\quad 0. Output is empty.\quad \hrulefill\null
\smallskip
\listing{stest0.tex}
\medskip
\noindent Test in line 10 is false.
\bigbreak
\noindent\hrulefill\quad 1. Output is {\tt SPACE EXPLICIT}\quad \hrulefill\null
\smallskip
\listing{stest1.tex}
\medskip
\noindent Test in line 10 is true. Test in line 18 is true. Test in line 26 is true.
\bigbreak
\noindent\hrulefill\quad 2. Output is {\tt SPACE FUNNY EXPLICIT}\quad \hrulefill\null
\smallskip
\listing{stest2.tex}
\medskip
\noindent Test in line 10 is true. Test in line 18 is false. Test in line 35 is true.
Test in line 37 is false.
\bigbreak
\noindent\hrulefill\quad 3. Output is {\tt SPACE}\quad \hrulefill\null
\smallskip
\listing{stest3.tex}
\medskip
\noindent Test in line 10 is true. Test in line 18 is true. Test in line 26 is false.
Test in line 52 is false.
Test in line 55 is false.
\bigbreak
\noindent\hrulefill\quad 4. Output is {\tt SPACE FUNNY}\quad \hrulefill\null
\smallskip
\listing{stest4.tex}
\medskip
\noindent Test in line 10 is true. Test in line 18 is false. Test in line 35 is false.
Test in line 52 is false.
Test in line 55 is false.
\bigbreak
\noindent\hrulefill\quad 5. Output is {\tt SPACE ACTIVE}\quad \hrulefill\null
\smallskip
\listing{stest5.tex}
\medskip
\noindent Test in line 10 is true. Test in line 18 is true. Test in line 26 is false.
Test in line 52 is false.
Test in line 55 is true.
\bigbreak
\noindent\hrulefill\quad 6. Output is {\tt SPACE ACTIVE}\quad \hrulefill\null
\smallskip
\listing{stest6.tex}
\medskip
\noindent Test in line 10 is true. Test in line 18 is true. Test in line 26 is false.
Test in line 52 is true.
\bigbreak
\noindent\hrulefill\quad 7. Output is {\tt SPACE FUNNY ACTIVE}\quad \hrulefill\null
\smallskip
\listing{stest7.tex}
\medskip
\noindent Test in line 10 is true. Test in line 18 is true. Test in line 35 is false.
Test in line 52 is false.
Test in line 55 is true.
\bigbreak
\noindent\hrulefill\quad 8. Output is {\tt SPACE FUNNY ACTIVE}\quad \hrulefill\null
\smallskip
\listing{stest8.tex}
\medskip
\noindent Test in line 10 is true. Test in line 18 is false. Test in line 35 is false.
Test in line 52 is true.
\bigbreak
\noindent\hrulefill\quad 9. Output is {\tt SPACE FUNNY ACTIVE}\quad \hrulefill\null
\smallskip
\listing{stest9.tex}
\medskip
\noindent Test in line 10 is true. Test in line 18 is false. Test in line 35 is true.
Test in line 37 is true.
\bye
