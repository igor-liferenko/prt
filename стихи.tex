%&12pt
\magnification=\magstep2
\advance\pdfvorigin by-2truecm
\advance\vsize by3.5truecm

\output={
  \ifnum\outputpenalty=-10001
    \footline={\hfil}
  \else
    \def\pagebody{
      \vbox to0pt{
        \font\dingbat=dingbat at10truept \dingbat
        \setbox0=\hbox{^^61}
        \dimen0=24\wd0 \advance\dimen0-\hsize
        \kern-.5cm
        \hbox{\kern-.5\dimen0^^61\hbox{\leaders\hbox{^^62}\hskip22\wd0}^^63}
        \nointerlineskip
        \vbox{\leaders\hbox{\kern-.5\dimen0^^64\hskip22\wd0^^65}\vskip 37\ht0}
        \nointerlineskip
        \hbox{\kern-.5\dimen0^^66\hbox{\leaders\hbox{^^67}\hskip22\wd0}^^68}
        \vss}
      \nointerlineskip
      \moveright.5cm \vbox to\vsize{\boxmaxdepth=\maxdepth
        \dimen0=\dp255 \unvbox255
        \kern-\dimen0 \vfil}}
  \fi
  \plainoutput}

\def\makefootline{\baselineskip=25pt
  \lineskiplimit=0pt
  \font\twelverm=omr9
  \line{\the\footline}}

\font\titlefont=omssdc10 at27.778pt

\raggedbottom
\parskip=0pt
{\catcode`\^^M=\active
\gdef\title#1^^M{{\bf#1}\par}
\gdef\date#1^^M^^M{\ifhmode\errmessage{j}\fi\hbox to10cm{\hss\it#1}}}
\def\par{\leavevmode\endgraf} \obeylines \let\par=\endgraf

\line{}
\vfil\penalty-10001

\line{}
\vfil\penalty-10001

\kern.25cm
\moveright.78cm\vbox{\titlefont \baselineskip=34.7pt
Есть Прелесть в\par Жизни Деревенской}
\vfil\penalty-10001

\kern.5cm

\title ЗИМОЙ В ДЕРЕВНЕ

У ёлок варежки из ваты.
Берёзки мёрзнут на юр\'у.
Ворон, синичек мир крылатый
Поближе тянется к двору.

Который день пурга гуляет,
Стучит крупинками в окно,
Живую влагу обращает
В полупрозрачное стекло.

Мороз снаружи. Зябко в доме.
Нарушен жизненный уют.
Где подевались те циклоны,
Которые тепло несут?

Приходит ночь со снежным валом,
Стучит, гудит в печной трубе.
И мёрзну я под одеялом
В моей остуженной избе.

\date Февраль 1988 г.

\eject

\kern-2.1cm

\title ЗА СЕЛОМ
\kern-7.2pt
За селом Солянкой ---
Колки да полянки.
Сколько глаз хватает ---
Тучные поля.
Жаворонки в небе,
Вольный ветер в хлебе.
Здесь и начинается
Родина моя.
\kern-7.2pt
Сразу на опушке
Слышен зов кукушки.
В ельнике меж кочек
Ручеёк журчит.
Запахи цветочные,
Звуки полуночные.
На меня вселенная
Звёздами глядит.
\kern-7.2pt
Лай собак в деревне,
В ветках пташек пенье.
Караси играют
В утреннем пруду.
Солнце и прохлада ---
Всё, что сердцу надо.
Не зовите в город,
В город не пойду.
\kern-7.2pt
Сразу за Солянкой ---
Колки да полянки.
От цветов и красок
Аж в глазах рябит.
Край ты наш привольный,
Мы тобой довольны.
Нам тобой гордиться,
Нам тебя любить.
\kern-12.5pt
\date Январь 1985 г.

\eject

\kern-.3cm

\title ВЕСЕННИЕ ЗАКАТЫ

Какие дивные закаты
У нас весною за селом!
А солнце кажется девчатам,
Как медь, натёртая песком.

Играют, бегают ребята.
И с ними тени их снуют.
Косые тени-великаны
Такую же игру ведут.

В лесу, где царство Берендея,
В цветах, где травка на лугу,
Под взглядом глаз твоих робею,
Но, что прикажешь, всё смогу.

Не каркай ты, треклятый ворон,
И пусть закроет хищник пасть.
Я за тебя отваги полон,
И волоску не дам упасть.

Какие дивные закаты.
Дрожат последние лучи.
Как в сказку, мысль несёт куда-то,
И я молчу\dots И ты молчи\dots

\date 1985 г.

\eject

\kern-.3cm

\title НА СЕНОКОСЕ

Хорошо на сенокосе!
Утомлённый, чуть живой,
Я бреду в шалаш к откосу
С родниковою водой.

День ушёл с рабочей песней,
Ночь крадётся на луга.
На погоду в поднебесье
Месяц выставил рога.

Чувства гордость растоптали.
Я гляжусь в твоё лицо.
Мне кузнечики сковали
Обручальное кольцо.

Я на всё с тобой согласен,
Коль душа лежит к душе.
С милой белый свет прекрасен,
С нею рай и в шалаше.

И усну опустошённый
Я со звоном комара.
Очарованный, влюблённый,
Чтобы сил набраться до утра.

\date 1986 г.

\eject

\kern-1cm

\title ТЫРБЫЛ

За деревней Спасовкой,
На ручье Тырбыл
Не однажды в праздники
Я с друзьями был.

Струи серебристые
Бьют из-под горы.
В тихих водах чистые
Видятся миры.

По весне черёмуха
Воздухом пьянит,
А вода без отдыха
По камням журчит.

По полям крутятся
Тучные стада,
В водопоях мутится
Чистая вода.

В заводях без камешков
Травы шелестят,
Уточка-хозяюшка
Сторожит утят.

Ниже по течению,
Где была Ловать,
Красную смородину
Едут люди брать.

\eject

\kern.2pt

Там поля клубничные,
По лесам грибы.
Берега приличные
Выставили лбы.

Где теченье быстрое,
Берега узки,
Там живые молнии ---
Рыбы харьюзки.

За деревней Спасовкой,
На ручье Тырбыл,
Отдыхая в праздники,
Я уху варил.

\date 1988 г.

\eject

\kern-1cm

\title НА РЫБАЛКЕ

Когда закончилась неделя,
И вы хотите отдохнуть,
Готовьте удочку немедля,
Спешите на гмирянский пруд.

И здесь, и там стоят машины.
Безвучно дремлют тростники.
Вдали у зарослей кувшинок
Сидят на лодках рыбаки.

Всплеснёт карась. И тихо снова,
Лишь по воде круги пойдут,
Но от события такого
У рыбаков сердца замрут.

Вот поплавок опять качнулся,
Подсечка! Удочка --- в дугу\dots
Сосновый бор уже проснулся,
На том чернеет берегу.

Лишь первый луч осветит воду,
В костре погаснет уголёк,
По чутким клавишам природы
Промчится сонный ветерок.

Плывет рыбак --- домой собрался.
И сразу разговор пошел:
--- Досадно, вот такой! Сорвался\dots
--- И у меня большой сошёл\dots

\eject

\kern.2pt

Вот осмотрел рыбак машину,
Сложил все снасти: стой - не стой\dots
Прошёл, ногой потрогал шину,
Завёл --- и тронулся домой.

Над лесом каркает ворона,
Повисли звуки над селом.
Сосна отряхивает крону,
И радость шепчется с добром.

\eject

\kern-.8cm

\title НОЧЁВКА У РЫБАЦКОГО КОСТРА

Хлеба нарезали всем по горбушке,
Плащ расстелив у костра на траве.
Водку распили из старенькой кружки.
Кровь разыгралась. Туман в голове.
Все нахлебались горячей ушицы.
В теле истома и липкая лень.
Лишь балагуры плетут небылицы,
Да в голове над сознанием тень.
Где-то кричит коростель на болоте.
Мирно гудят комары.
Ждите хоть год, никогда не найдете,
Лучше, чем эта, поры.
Гаснет костёр. Только искры метутся.
Хочется спать\dots{\spacefactor=3000} Да какой там уют?!
Дивы ночные за спиной крадутся,
Схватят, с собой унесут.
Сон одолел. В разных позах уснули.
Пала роса на траву.
Ночью озябли ребята. Проснулись,
Тянутся ближе к костру.
Одеяло утро сбросило.
Темнота раскрыла даль.
Небеса закупоросило.
В небытье ушла печаль.
Расплескался свет по воле.
Счастьем вспыхнула душа.
Летом русская природа
До чего же хороша!

\eject

\kern6pt

Птичьи трели разливаются.
И в реке вода журчит.
Вальсом воздух наполняется,
Вороньё в лесу кричит.
Солнце выше поднимается.
День рабочий настаёт.
Кто-то с кем-то разлучается,
А кого-то кто-то ждёт.
Снасти сложили в машину.
Сели, тронулись домой,
А дорогой порешили\dots
Снова ехать в выходной.

\eject

\kern-1cm

\title К 70-ЛЕТИЮ РАЙОНА

Между Канской землёй и Уярской
Расположился наш район.
По законам властей пролетарских
Назван Рыбинским и утверждён.

На Московской дороге бескрайней,
От железных дорог вдалеке,
Село Рыбное стало центральным,
Что стояло на Рыбной реке.

Кто пешком, кто богаче --- на дрогах,
Прихвативши с собой ``сидорки'',
По разбитым сибирским дорогам,
Добирались туда мужики.

\leavevmode\llap{``}До чего неудобственно, паря, ---
Говорил мужик мужику, ---
Время нет, а неделями тратишь
По малейшему пустяку.

Вот железной дорогой удобнее
И быстрей, и короче пути.
Есть ведь Троицко-Заозёрное,
Центр бы туда перенести.''

Так и сделали умные люди.
Сразу жизнь веселее пошла.
Поездами добраться нетрудно
Из любого глухого села.

\eject

\kern.2pt

Заозёрный --- большая деревня.
Как во всякой сибирской глуши,
По пригоркам стояли деревья,
По болотам росли камыши.

Годы шли, и леса вырубались.
Отходила всё дальше тайга.
Хорошело село, разрасталось,
Истощалась, мелела Барга.

Изобильный район и богатый,
Очень важный для жизни страны.
Люди в нем работящие, знатные,
Как в труде, так на поле войны.

Добывается уголь в разрезах.
Созревают хлеба на полях.
На лугах стада тёлочек резвых,
И в достатке зерно в закромах.

Заозёрный теперь уже город ---
В сорок пять исторических лет.
Среди прочих российских он молод
И трамвая пока ещё нет.

Но я верю, со временем будет,
Что для города быть суждено.
В нём живут настоящие люди,
И с прогрессом они заодно.

\eject

\kern-1.3cm

\title В РОДНОЙ ДЕРЕВНЕ

Есть прелесть в жизни деревенской,
Свои особые черты
В людской любви и ласке женской.
Я здесь с природою на ``ты''.

Нависли ивы над водою.
Паук для мухи сети вьет.
Связал два облака чертою
Летящий в небе самолёт.

Гусак шипит, гогочет стая,
Петух подрался и поёт.
Из-под ворот собака злая
Кому-то ``интервью'' даёт.

Ликует жизнь, исчезла злоба.
Мать ребятне наказ даёт:
\leavevmode\llap{``}Играйте, да смотрите в оба ---
Чёрт за углом проступка ждёт''.

Когда тоска одолевает,
Пойду по улице пройду\dots
Я знаю всех, меня все знают ---
Здесь каждый житель на виду.

Ходил, летал, бывал на море ---
Таков уж наш двадцатый век.
И только здесь я что-то стою,
Лишь здесь я --- нужный человек.

\date 1992 г.

\eject

\kern-2.04cm

\title ПАМЯТИ Е. И. МУХОРТОВА

Давайте вспомним и о нём.
На стол блинов поставим горку
И добрым словом помянём,
Как жил наш друг Е.И.Мухортов.
Он был директором ДК,
Работою своей гордился.
Дни напролёт и вечера
Среди народа находился.
У женщин слыл за чудака:
До дней последних не женился.
Среди мужчин --- за простака:
Бесплатно, ревностно трудился.
Весной с лопатою ходил,
Рыл землю и готовил ямки.
Потом деревья в них садил
На улицах родной Солянки.
Минули годы, лес растёт.
Весною птицы прилетают.
Теперь Мухортова уж нет,
Но сделанное им не умирает.
Пройдись по улицам села:
Одна другой светлей и краше.
Смолистый дух даёт сосна,
Косынками берёзки машут.
И кто бы что ни говорил,
Народам ведано издревле:
\leavevmode\llap{``}Тот не напрасно век прожил,
Кто для людей сажал деревья''.
Мы перед памятью в долгу,
Портрет в ДК повесим в рамке
И надпись сделаем к нему:
\leavevmode\llap{``}Почётный гражданин Солянки''.
\kern-7pt
\date 1988 г.

\eject

\kern-.5cm

\title КАФЕ

По селу плывут слухи разные
И несут они вести важные.
Сватье сват брюзжит, распинается:
--- Молодёжь у нас дурью мается.

Пусть потешатся злопыхатели,
А что скажут нам доброжелатели?
--- Молодежь у нас не ленивая,
Беспокойная и счастливая.

Что ни юноша --- первым кажется.
Что ни девица --- то красавица.
Сам комсорг-вожак с комсомольцами
Подвизались быть добровольцами.

Они сделали всё возможное,
Чтоб открыть кафе молодёжное.
Дни до полночи мыли, красили
И в труде своем были счастливы.

Заходи, народ, волей вольною
В наш салон-кафе безалкогольное.
Распивай чаи, кушай пряники,
Хмурый зимний день станет праздником.

\eject

\kern-2cm

\title ЖЕНЩИНАМ

Весну я встречаю, как счастье,
В неё, видно, все влюблены.
Конечно, бывают ненастья,
Но чаще погожие дни.
Белее и чище берёзы,
Заметно прибавились дни.
Ещё не ослабли морозы,
Но видится поступь весны.
У крыш кружевные сосульки,
Шустрее снуют воробьи.
А утром хрустят в переулке
Поспешные чьи-то шаги.
И бабушки наши, и мамы,
Подруги, что с нами живут,
Влюблёнными смотрят глазами,
Подарков и праздника ждут.
Я вам по секрету открою:
Товар на прилавках лежит,
Но хочется чего-то такого,
Чтоб радостно было дарить.
Все добрые женщины --- милы.
Чужие, свои --- всё равно.
Жалею когда-то счастливых,
Покинутых кем-то давно.
Пусть радостно светятся лица,
Серебряный сыплется смех.
Пусть горе уйдёт небылицей,
И счастье прибудет для всех.
Целую вас нежно и жарко ---
Друзья не сочтут за грехи.
Пусть будут скромным подарком
Душевные эти стихи.
\kern-7pt
\date Март 1986 г.

\eject

\kern-1cm

\title НЕОБХОДИМЫЙ ЧЕЛОВЕК

Морозным днём иль в летний зной,
Где рельсы тянутся по шпалам,
Стучит ключом или киркой
Рабочий Ваня Шаповалов.
Монтёр, как все, работу знает,
Но есть особые штрихи:
Завидным даром обладает
Из слов монтировать стихи.
Куда он думы устремляет?
Куда ведёт его звезда?
Холодным ветром обдавая,
Проходят с шумом поезда\dots
На уши давит близкий грохот,
И рельсы гнутся и дрожат.
Из-под колёс, рифмуя строки,
К нему метафоры летят.
В словесном вихре мысль порхает,
Он ловит трепетной рукой.
И никого не замечая,
Строфу черкает за строфой.
Светлей становится повсюду,
И радостней тяжёлый труд.
Прекрасно, что такие люди
Здесь с нами рядышком живут.
В тяжёлом грохоте металла
Кончается двадцатый век.
И в этой жизни Шаповалов ---
Необходимый человек.

\eject

\kern-1cm

\title ПОСЛЕ ДОЖДЯ

Дождик кончился. Тепло.
На дорожках лужицы.
А водичка в ручейках
Пенится и кружится.
Мы --- ребята-моряки,
Не боимся мы реки.
Мы --- ребята смелые,
По водичке бегаем.
Мамочка стучит в окно,
Бабушка ругается ---
Это внучка у неё
Грязнулей называется.
Только папа не бранит,
Папа улыбается:
И девчонке смелой быть
Тоже полагается.
Мне нисколечко не больно,
Я упала только раз.
Если б были все довольны,
Я б ещё похлюпалась.

\eject

\kern1cm

\title БАБУШКА И КОЗЫ

Козочки у бабушки
Весело живут.
Щиплют вволю травушку,
Молоко дают.
Мы с братишкой рядышком
К бабушке пойдем.
Поедим оладушек,
Молочка попьём.
На лужайке с бабушкой
Козочек пасём,
Посидим с ней рядышком,
Песенки споём.
Любим сказки страшные ---
Их не перечесть.
Хорошо, что бабушки
В нашей жизни есть.

\eject
\kern.4cm
\moveright1.13cm\vbox{\titlefont \baselineskip=34.7pt
То Было\par Трагедией Века}
\vfil\penalty-10001

\kern-2.1cm

\title СОЛЯНСКИЙ БАТАЛЬОН

Их было три сотни, почти батальон,
Солдат из деревни родной.
На них похоронки разнес почтальон.
Тела их в могилах давно.

То были все парни и мужики.
Деревня без них опустела.
Остались солдатки и старики,
Да малые дети в постелях.

Под Ржевом зимою в пургу и мороз
Сибирские Гришки и Ваньки,
3акаменев от обиды и слёз,
Шли с кулаками на танки.

Тогда их немного осталось в живых:
Давили немецкие танки.
Всю мощь огневую нацелили в них---
Солдат из деревни Солянка.

Нет-нет, да и вспомнят родные о них.
Висят фотографии в рамках,
Да старые письма о днях фронтовых
Мужчин, так любивших Солянку.

Давайте наполним стаканы вином
В день нашей великой Победы,
Помянем погибших и вспомним о том,
Как прожили мы эти беды.

Мы жизнью обязаны тем, кто погиб.
Пусть памятник напоминает
О тех, кто в далеких могилах лежит.
Их помнят, а, вспомнив, рыдают.
%
\eject

\kern-2.2cm

\title ПАМЯТНИК
\kern-8.5pt
Пускай здесь не было боёв,
Не прорывались вражьи танки.
В одном из лучших уголков
Воздвигнут памятник в Солянке.
\kern-8.5pt
За плечи вскинут автомат,
И с обнажённой головою
Стоит, задумавшись, солдат---
Он только вышел с поля боя.
\kern-8.5pt
Никто не скажет точно, чей он.
А старожилы говорят:
\leavevmode\llap{``}Один из Калиниченко
Или из Котовых ребят\dots''
\kern-8.5pt
И кто такой на пьедестале?
Уж мало тех, кто помнит их.
\leavevmode\llap{``}У этих-то двоих не стало\dots
А вот у Котовых --- троих\dots''
\kern-8.5pt
И всё равно не обознались.
Пусть будет этот или тот.
Всего и Солянку не вернулось
Примерно около трёхсот.
\kern-8.5pt
Молчит солдат, и всё смолкает.
Всплывают память и мечты.
Молодожёны тут бывают,
И дарят девушки цветы.
\kern-8.5pt
Пусть он недвижен, пусть молчит.
Солдат, сержант --- суть не в званьи,
Погибшим в память он стоит,
А всем живущим --- в назиданье.
\kern-8.5pt
А где-то и сейчас бои
Снаряды рвутся, мчатся танки,
И памятник у нас стоит
Возле ДК среди Солянки.
%
\eject

\kern-1cm

\title МОЙ ОРДЕН

Мне орден лучистый вручила страна.
В нём надпись словами по кругу:
ОТЕЧЕСТВЕННАЯ ВОЙНА.
Я был среди тех, кто за Родину встал,
Когда враг жестокий коварно напал.
Коробился, гнулся металл.
Воронки, окопы, блиндажи\dots
В разрывах снарядов, под танком лежи,
В атаку вставай, не дрожи.
Отцы наши там погибли, друзья.
Пусть будет им пухом родная земля.
И чудом остался лишь я.
И вот за себя и за них я живу,
Работаю честно и мир сторожу.
И орден по праву ношу.

\eject

\kern-1cm

\title НА ВСТРЕЧЕ ``ОГНЕННЫЕ ДОРОГИ''

В деревне с рассветом поют петухи,
Заботы ложатся на плечи.
Я маюсь всю ночь, сочиняю стихи,
Пишу о торжественной встрече.

Девчонки и парни --- студенты они,
А это великая силища.
Решили собрать ветеранов войны ---
Выпускников педучилища.

Когда над страной разразилась война,
Никто не остался в сторонке.
Всех взрослых мужчин подбирала она
И слала взамен похоронки.

3а Каном-рекою, в густом сосняке
Отцы становились солдатами.
И канули в вечность, как слёзы в реке,
Которыми матери плакали.

На встрече сидели одни старики
И юность свою вспоминали,
Как в тяжкие годы военной поры
И мёрзли, и голодали.

Несчётное множество день ото дня
Критических было моментов.
Надёжною выручкой было тогда
Великое братство студентов.

\eject

\kern-1cm

Солидные дамы --- девчонки в те дни
В фуфайках одетые спали.
В нетопленых классах учились они,
На книжках, газетах писали.

А эта!\dots{\spacefactor=3000} Я к ней подступиться не смел:
Глядела надменно и косо
На парня, что сзади за партой сидел,
И трогал за длинные косы.

Ребята, что были на фронте тогда,
Такие, как внуки сегодня,
Теперь --- ветераны войны и труда,
Высоких наград удостоены.

В той группе смеются, а в этой грустят,
Платками глаза вытирают.
Друзей и любовь не воротишь назад,
А что впереди ожидает?\dots

Уже наступает предутренний час,
А мы даже глаз не сомкнули.
Спасибо, ребята, за то, что вы нас
Опять в нашу юность вернули.

Пусть жизнь ваша будет
Без трещин и ран.
Учитесь, трудитесь без лени.
Я слышал, когда говорил ветеран:
\leavevmode\llap{``}Для нас вы --- достойная смена''.

\date Март 1985 г.

\eject

\kern-2.2cm

\title ДРУГУ, ВЕТЕРАНУ, УЧИТЕЛЮ \endgraf %
\ \ \ \ \ \ \ \ \ А.Д. МАКАРЕНКО
\kern-4pt
И до войны ты был учитель,
Лишь старше стал, да больше стаж.
Наш воин --- скромный победитель,
А для Отечества --- солдат.
\kern-4pt
Ты в праздник надеваешь китель.
До блеска чистишь ордена.
И порассказывать любитель
За стопкой крепкого вина.
\kern-4pt
Картечь спасала батарею.
Враги трезвели, шли назад.
И грозен был в дыму сражений
Пехоты нашей резкий мат.
\kern-4pt
Не покидая батарею,
И при разрывах вражьих мин,
Освобождал Варшаву, Вену,
А в мае штурмом брал Берлин.
\kern-4pt
Судьба тебя не баловала,
А всё-таки тебе везло.
Из трёх полученных ранений,
Два оказалися легко.
\kern-4pt
Всё время подвергаясь смерти,
Почти не спал, не часто ел.
Таких, как ты, не брали черти,
Ты выжил, ты остался цел.
\kern-4pt
Учитель, как и был учитель\dots
Судьбой доволен, жизни рад,
Наш воин --- скромный победитель,
А для Отечества --- солдат.
\kern-11pt
\date 1988 г.

\eject

\kern-1cm

\title ДЕДУШКА И ВНУКИ

Из садика вечером внуки пришли
И деда упрашивать стали:
\leavevmode\llap{``}Ты нам о войне, о себе расскажи,
За что ордена и медали?''

А дедушка внуков руками обнял,
Стал грустным, подумал немного,
Вздохнул глубоко и тихонько начал:
\leavevmode\llap{``}Я вам не желаю такого.

Когда над страной разразилась война,
Мы в старшие классы ходили.
Тяжёлые были тогда времена,
И все в напряжении жили.

От взрывов и крови стонала земля
От Белого моря до Чёрного.
Сражались солдаты, держалась Москва,
И делали всё невозможное.

А матери наши в полях, у станков,
Работали и голодали,
Мужей заменив, а мальчишки --- отцов,
Для фронта победу ковали.

Не год и не два, а четыре прошло,
И мы в своё время успели
В атаки сходить. Только нам повезло:
Остались ранения в теле.

\eject

\kern-1cm

Пятьдесят миллионов людей
В войне этой в землю упрятаны.
Неужто найдётся, как Гитлер, злодей,
Который мечтает об атомной? \dots

Идите, ребятки, а я посижу.
Ложитесь и спите спокойно.
А я, если надо, ещё послужу,
Чтоб сгинули всякие войны.''
\kern-4pt
\date Февраль 1986 г.

^^M\title ДЕТИ ВОЙНЫ

Не мерить кровь и слезы тоннами\dots
Кому и сколь пришлось страдать.
И на машине электронной
Статистикам не сосчитать.

Мы шли на фронт землёй Смоленщины,
Минуя минные поля,
Участливо глядели женщины
В освобождённых деревнях.

То не деревни --- лишь названия,
Испепелённая земля\dots
Стояли печки --- изваяния
В местах, где улица была.

И нам, недавним малолеткам,
Был за отцами путь один\dots
Но мы шутили не по-детски:
\leavevmode\llap{``}Где тут дорога на Берлин?''

\eject

\kern-2cm

Бой первый был последним боем
Для тех ребят, что полегли.
Они не встанут перед строем,
Оставшись в памяти живых.

Оглохший в грохоте разрывов,
Я по врагу стрелял.
Потом у вражеских окопов
В атаке раненный упал.

И в ППГ (была конюшня),
Брезент подо мной от крови мок,
А слева малая девчушка
Лежала без обеих ног.

Тупая боль тяжёлой раны
Сверлила мозг, лишила сна,
Но к девочке-малютке жалость
Сильнее боли сердце жгла.

Где тот фашист, что бомбу сбросил?
Он продолжает убивать\dots
Его б сюда, к столбу конюшни,
Перед малюткой привязать.

И, может, там, на земле смоленской,
Войной порушенном краю,
Живёт старушка на протезах,
Ей не забыть про ту войну.

И если ты осталась, выжила
И дожила до этих лет,
Знай, помню о тебе, родимая,
И шлю сибирский свой привет!
\kern-3pt
\date Февраль 1985 г.

\eject

\kern-1cm

\bgroup %
\font\bf=omb10 at 12pt %
\title РОДСТВЕННИЦЕ ПОГИБШЕГО СОЛДАТА
\egroup %

Ты с укоризной не гляди
Такой уж жребий нам достался:
В ужасном крошеве войны
Один погиб, другой --- остался.

Я ж знаю, что не виноват.
Пришедший с фронта невиновен,
Что не вернулся твой солдат,
Убит и где-то похоронен.

Мне тоже нелегко пришлось.
Скажу тебе (боюсь обидеть),
То, что мне в жизни довелось,
Дай бог, тебе того не видеть.

Был слышен плач из всех углов.
Рыдала мать, сестрёнки выли.
Прибавилось сирот и вдов ---
Мы похоронку получили.

И замер я, без слёз стоял,
Ко внешней жизни безучастный.
А дед-сосед: ``Крепись, --- сказал, ---
Ты не один такой несчастный''.

Я упросил не провожать:
Не мог я видеть слёз у мамы,
Когда пошёл в военкомат,
Повестку комкая в кармане.

\eject

\kern-1cm

А смерч военный бушевал\dots
В атаку поднялась пехота.
Я, как подкошенный, упал ---
Бил вражий пулемёт из дзота.

За днём ушедшим ночь идёт.
И, может быть, опять во мраке
Рукой сотру горячий пот,
Побыв в приснившейся атаке.

Уже не малый срок лежит
Меж днями этими и теми.
Я не могу переносить
Кинокартин военной темы.

И ужас сердце леденит,
Когда подумаю такое:
Что может стать\dots{\spacefactor=3000} что может быть\dots
Что мир непрочен под луною.

И ты сейчас не зря живи,
Во всяком случае, старайся,
Чтобы хоть внук твой не погиб,
А, отслужив, живым остался.

Ты с укоризной не гляди
На тех, чей возвратился воин.
Ведь твой солдат --- в сердцах живых,
Великих почестей достоин.

\date 1984 г.

\eject

\kern-1cm

\title СОЛДАТСКАЯ ПОЭМА

Давно закончилась война,
Замолкли грозные раскаты.
Ушли ни с чем в свои края
Разгромленные супостаты.

И только шепчутся юнцы,
Когда в субботу в жаркой бане
Увидят рваные рубцы
На старом теле ветерана.

Да грустно бабушки вздохнут,
Сплетая внучкины косички.
О юности своей всплакнут,
О доле девушки-медички.

Я знаю многих, сам такой,
Когда невзгоды вспоминают,
Прикрывшись старческой рукой,
В платочек слёзы проливают.

Подступит к горлу ком тугой,
Сожмётся сердце, боль тревожит.
И только чистою слезой
Из покрасневших глаз исходит.

Всплывает, что легло на дно,
В крутнувшемся водовороте.
Как кадры старого кино,
Я вспоминаю войны эпизоды.

\eject

\kern-1cm

Чередовались взрывы воем
Немецких мин смертельно злых.
В померкшем дне изрытым полем
Гнал ездовой своих гнедых.

На кочках прыгали колёса,
Метался в стороны фургон.
А в адрес бога в небо нёсся
Солдатский мат и жуткий стон.

В фургоне, кроме ездового,
Солдаты грязные в крови.
Один держал руками ногу,
Другой с осколками в груди.

Остановились за стеной.
\leavevmode\llap{``}Ну, что, братишки, все вы целы?'' ---
Оборотился ездовой,
Когда ушли из-под обстрела.

Потом колёса тарахтели
По утрамбованной земле.
Палило солнце. Пить хотелось
В полузабытье, в полусне.

Вчера деревню взяли с боем ---
Один остался целым дом.
На пол траву стелили слоем,
Санрота разместилась в нём.

И мы, как снова народившись,
Раскрыв от удивленья рты.

\eject

\kern-1cm

Ты перед нами появилась ---
Предвестник жизни и мечты.

С крыльца заботливо сбежала
Девчушка с чёлкою на лбу.
Нам слезть с телеги помогала,
Укладывала на полу.

Чтоб стон сдержать, кусал я губы,
До пепла жёг огонь в груди.
\leavevmode\llap{``}Что больно, родненький? Мой любый,
Ты уж немножко потерпи.''

С засохшей кровью отрывала
От ран солдатское ``хэбэ'',
Над каждым долго колдовала
В йодом пахнущей избе.

Теплилась жизнь в ослабшем теле\dots
Уж очень нам хотелось жить.
Но самолёты налетели,
Деревню начали бомбить.

Земля качалась, бомбы рвались.
И стёкла падали на нас.
А ты у косяка стояла,
И слёзы капали из глаз.

Мы знали, что ты жизнь любила,
Иначе и не может быть.
В твоих слезах такая сила ---
Беду сумеет отвратить.

\eject

\kern-.5cm

Вдруг грохот смолк и тихо стало.
Нас распирала тишина.
Еще разочек попугала,
Но жить оставила война.

Вот ты согнулась, наклонилась,
Повязки стала поправлять.
С любовью той, что мне не снилась,
Хотелось руки целовать.

С душой тогда кристально чистой
Я этого не мог посметь.
Тогда, еще почти мальчишка
В неполных восемнадцать лет.

Доволен тем, что жив остался,
Преодолевши смерть и страх.
Почти полгода поправлялся
Я в тыловых госпиталях.

И там, у жизни на краю,
Со смертью бой вели сестрички.
Спасали раненых в бою
Врачи и девочки-медички.

Хочу прощения просить
Сквозь толщу лет, что не воюем,
И ваши руки осенить
Хотя б воздушным поцелуем.

\eject

\kern-1.5cm

\title МОЁ ПОКОЛЕНИЕ

{\font\eightrm=omr8 \eightrm \baselineskip=10pt%
Мы жили ожиданием светлого будущего.
Теперь живём ожиданием последнего часа.\par}%

Чем дальше военные годы,
Тем старше становимся мы.
И видим сквозь время невзгоды
Чудовищно страшной войны.

Вернуться мог только калекой,
А целый --- обратно в бой.
То было трагедией века,
То было российской судьбой.

Победа давалась не просто,
Мы технику брали в штыки.
И тлеют солдатские кости
От Волги до Эльбы реки.

Работали честно, старались,
Но лучше не начали жить.
Нас в рабстве колхозном держали,
Вождей заставляли любить.

И если кто думал иначе,
То он --- затаившийся враг.
Постонет, повоет, поплачет
Переселённый в ГУЛАГ.

Безвинных под пыткой ломали,
Садистский гулял беспредел,
И каждую ночь вызывали
Из камер во двор на расстрел.

\eject

\kern.2pt

Потом началась перестройка,
Подорожали гробы.
Свобода, долги, неустойки ---
Мы стали чужие рабы.

И дети, рабы капитала
(Лукавый политик не лги)
Внучей наших тоже продали
За взятые Вами долги.

Россия, родная Россия,
Где наша с тобою земля ---
Славян обездоленных, сирых,
Таких безутешных, как я?

\eject
\kern.4cm
\moveright1.825cm\vbox{\titlefont \baselineskip=34.7pt
Школа ---\par Моя Судьба}
\vfil\penalty-10001

\kern-1cm

\title НОВАЯ ШКОЛА

Нет, не в сказке про Емелю,
Там, где щука говорит.
Здесь у нас, в родной деревне,
Чудеса народ творит.

Настоящий, не бумажный,
Не какой-то теремок ---
Вырос корпус трёхэтажный
К дню занятий, точно в срок.

Нам строителей, ребята,
Есть за что благодарить:
Их трудами, их заботой
Школа новая стоит.

В этих свежих, чистых классах
Много света и тепла.
В мир науки, в мир прекрасный
Поведут учителя.

Беспокойный день стараний
Был родителям вчера:
Поприветствовать в День Знаний
Вас готовилась страна.

\eject

\kern.2pt

И сейчас стоите гордо
Вы с букетами в руках.
Чтоб, окончив школу, твёрдо
Встать на собственных ногах.

Сколько тайн, немых вопросов
У пришедших в первый класс\dots
Может где-то Ломоносов
Затерялся среди вас?

Так растите и учитесь
Час за часом, день за днём.
Мир и радость, вы учтите,
Добываются трудом.

Всё для вас, родные дети:
Ширь полей, сиянье дня,
Мир и счастье на планете,
Именуемой Земля.

\date Сентябрь 1985 г.

\eject

\kern-1cm

\title ВЕЧЕР ВСТРЕЧИ

Ничего, что зима на дворе.
Хорошо, что за окнами вечер.
В день субботний, опять в феврале,
Мне мила долгожданная встреча.

По весне возвращаются птицы,
Тянет, манит родная земля.
Вот и я прихожу поклониться
Тебе, школьная юность моя.

Я с отрадным волненьем гляжу
На почти позабытые лица.
И всё больше я в них нахожу,
Что с годами успело забыться.

Все так мило, что было тогда,
Как мы нежно друг друга любили.
Но всему наступает пора,
Аттестатами нас разлучили.

Разлетелись по разным местам ---
Кто куда, как пугливые птицы.
Для того, чтобы взрослыми стать,
Чтобы в жизни чего-то добиться.

Наша память вернёт иногда
Эпизоды из детского мира,
Как наивны мы были тогда,
Бескорыстно друг друга любили.

\eject

\kern-1cm

Можно много и страстно писать,
Но закончу такими словами:
Очень хочется всех вас обнять,
Как я рад, что увиделся с вами.

Ночь прошла, и рассвет со двора
В окна стылые тихо крадётся.
Расстаемся, и будет пора,
Когда встретиться снова придётся.
\kern-5pt
\date 1986 г.

^^M\title ШКОЛЬНИКАМ О ПУШКИНЕ

Всем мальчишкам и девчушкам
Я хочу сказать о том:
К нам в сознанье входит Пушкин
С материнским молоком.

\leavevmode\llap{``}Буря мглою небо кроет,
Вихри снежные крутя\dots''
Мать, качая, утешает
Неуёмное дитя.

Детство школьное подходит,
И ребёнок сам прочтёт,
Как балда чертей обманет,
У попа расчёт возьмёт.

Под горой не так уж дальней
Тайный ход, свеча горит.
Гроб качается хрустальный,
А в гробу царевна спит.

\eject

\kern-.5cm

Сказки радость представляют,
Хоть и знаешь наперёд:
Правда кривду побеждает,
Справедливость верх берёт.

С богатырской дивной силой,
Добывая честь в бою,
B путь с Русланом и Людмилой
Устремляю мысль свою.

И, гусаром оскорблённый,
Над рассказом слёзы лью,
Как смотритель станционный
Тщетно ищет дочь свою.

Вот на палках-шпагах храбро
Бьются школьные друзья.
Погоди же, глупый Швабрин,
Доберусь я до тебя.

Старшекласницы-Татьяны
Гордо смотрятся в трюмо.
И строчат тайком от мамы
Для ``Онегина'' письмо.

Русский дух убить не можно,
В гроб забвенья не сложить.
Вечно в памяти народной
Пушкин жил и будет жить.

\date 1987 г.

\eject

\kern-1cm

\title МОИМ УЧЕНИКАМ

Я строго вам в глаза гляжу,
И если бы вы только знали,
Как беззаветно вас люблю,
Вы б мне на нервах не играли.

Не скоро зазвенит звонок,
И хлынет детвора из классов.
Ну, а пока идёт урок,
Как самолёт своею трассой.

Я вам толкую и пишу.
Вы слушаете, сложив руки.
Тропою знаний вас веду,
Так мы грызём гранит науки.

Среди житейской кутерьмы
Пройдут года, минуют даты.
А, может, вспомните и вы
Урок, полученный когда-то\dots

Спешите слушать и смотреть.
И удивляться ежечасно,
Чтобы потом не пожалеть
О годах, прожитых напрасно.

Уже кончается урок.
Я ставлю в дневники отметки.
И долго не звенит звонок\dots
И шепчется сосед с соседкой\dots
\kern-3pt
\date 1984 г.

\eject

\kern-1cm

\title СТАРЫЙ ТОПОЛЬ

Красивая, новая школа
По новым законам живёт.
Где старая школа стояла,
Там тополь высокий растёт.

Стоит выше всех одиноко,
Как будто вокруг никого\dots
И ветер осенний жестоко
Терзает одежду его.

Шумит он и машет ветвями,
Порывы сильней и сильней.
А он свои листья теряет,
И с каждым --- тоскливей\dots{\spacefactor=3000} больней\dots

Но были весенние ночи
И тёплые летние дни.
Он хочет того иль не хочет ---
Со временем скрылись они.

Ему посвящаю стихи я,
В природе и в жизни борьба:
Ему не подвластна стихия,
Как мне не подвластна судьба.

Старею. Все чаще болею,
Но так же, как воин в бою,
Позиций покинуть не смею,
И раненный насмерть, стою.
\kern-3pt
\date Март 1985 г.

\eject

\kern-1cm

\title ПИСЬМО БЫВШЕМУ УЧЕНИКУ

Мой юный друг на флоте служит.
Подходит праздник, я спешу.
Пусть только вовремя получит ---
Ему послание пишу.

Туда, откуда день приходит,
Где начинается рассвет,
Где вал за валом волны ходят.
Я шлю, дружок, тебе привет.

Я не был там, но твердо знаю,
Проводишь время ты не зря.
Вот почему тебе желаю
Здоровья, счастья и добра.

Пусть в этот праздник новогодний
Среди товарищей, друзей
Тебе икнётся ненароком ---
Знай, я подумал о тебе.

Пусть не покажется случайным,
Хоть близко знал с десяток лет
Меня учителем, но тайно
В душе и мыслях я --- поэт.

Я ставил в дневники отметки
Своей ``карающей'' рукой.
А после сам болел нередко
По-детски чувственной душой.

\eject

\kern.2pt

Ни перед кем не пресмыкаюсь
И на работу не спешу.
Сижу на пенсии и маюсь,
Стихотворения пишу.

Живу в деревне и не каюсь.
Без громких званий и имён.
Как лягу, так и просыпаюсь,
Всегда Сафроненков Семён.

\date 1986 г.

^^M\title РАССТАВАНИЕ

Все закончены науки.
Годы позади.
Неизбежная разлука
Ждёт вас впереди.

Отзвенит звонок последний
В этот день весной.
Вы уйдете в мир безбрежный
Всяк своей тропой.

И потом, в грядущей жизни
Вспомните не раз,
Как гранит науки грызли,
И любимый класс\dots

\eject

\kern.2pt

О друзьях, с кем вместе были
Много зим и лет,
Горы радости делили
И монбланы бед.

Путь нелёгкий, путь тернистый
Ждёт вас впереди.
Выбирай дорогу чище
И смелей иди.

Счастья, бодрости, здоровья
Мы желаем вам.
Будущее строить
Помогите нам.

Школьный курс наук окончен ---
Все запасы впрок.
И в последний час урочный
Пусть звенит звонок.

\date 1993 г.

\eject

\kern-2.1cm

\title ТАНЯ

Вся она из чудной сказки,
Из фантазий, из мечты.
Где найти слова и краски
Описанья красоты?

Сколько в ней тепла и света!
Добродушна и мила
Таня, девушка-студентка,
Из соседнего села.

Нет её --- от скуки вянешь.
И стараешься найти.
На неё случайно взглянешь,
Глаз не сможешь отвести.

К Тане я неравнодушен.
Но признаться должен вам.
Этот лакомый кусочек
Мне никак не по зубам.

Слово ``Таня'' мне не ново,
Дорожу им с первых дней,
Потому что слово ``Таня'' ---
Имя матери моей.

И, наверное, ей не в радость ---
За приветом шлю привет.
Даже если с нею рядом:
Таня здесь и Тани нет.

Невозможно не влюбиться,
Жалко, возраст мой не тот.
И как в басне говорится
\leavevmode\llap{``}Видит око, зуб неймёт.''
%
\eject

\kern-1cm

\title ОПЕРИЛИСЬ СКВОРУШКИ

Оперились скворушки,
Выбрались на волюшку.
Под лучистым солнышком
Свой приют нашли.
Повзрослели мальчики,
Повзрослели девочки.
Стены классные тесны,
Школа позади.
До свиданья, мальчики!
До свиданья, девочки!
Наши судьбы разные ---
Что там впереди?
Есть друзья хорошие,
Есть подружки добрые,
Люди есть разные ---
Выбирай, гляди.
Не забудут скворушки
О родной сторонушке,
Будут помнить гнёздышко
Счастья и любви.
Оперились скворушки,
Выбрались на волюшку.
Будут, будут гнёздышки
Счастья и любви.

\date 1993 г.

\eject

\kern-1cm

\title СВАДЬБА

Ленты алые и голубые.
Люди старые и молодые.
И кому-то не горько вино.
То, что было недавно, --- давно.
Рос мальчишка, девчонка росла ---
Птица счастья на крыльях несла.
Не знали друг друга. Он и она.
А теперь это будет семья.
Крякнет, выпив, мужик: ``Хорошо!''
Ну, а матери-то каково?
Слёзы счастья, в душе пустота ---
Отрывают от сердца дитя.
Ленты алые и голубые,
На машинах шары росписные.
У дороги соседки стоят
И на свадебный поезд глядят.
--- Он-то, парень, собою хорош!
--- А она! Разве лучше найдёшь?
Беспристрастной здесь нет ни души.
--- Видно, оба они хороши!
Жили отдельно. Он и она.
А теперь это --- семья.
Пусть с любовью и в счастье живут
И с посылками аистов ждут.

\date Январь 1989 г.

\eject

\kern-1cm

\title ДЕНЬ ВОСПОМИНАНИЙ

Шли годы жизненных скитаний,
Но мы смогли любовь сберечь.
Пришёл наш день воспоминаний
И вечер долгожданных встреч.

Нам не забыть тех дней прекрасных,
Где было всё наверняка.
От первых дней, что в первом классе,
И до последнего звонка.

И только за порогом школы
Узнали мы, как трудно жить,
Что каждый шаг даётся с боем,
Непросто верить и любить.

Учителя, как наши няньки,
Вы дали всем нам, что смогли.
Вот в этом здании в Солянке,
Где годы детские прошли.

He очерствели наши души,
Хоть жизнь бросала вверх и вниз,
Раз мы хотим друг друга слышать,
Раз здесь сегодня собрались.

Ребята, кажется, звонок!
Где были вы так много лет?
Тогда мы все моложе были,
Не все мы здесь, кого-то нет\dots
\kern-6pt
\date 1993 г.

\eject

\kern-1cm

\title ПОСЛЕДНИЙ ЗВОНОК

Звенит звонок. Звонок последний
На заключительный урок.
На сердце грусть от трели медной,
Цветы рассыпались у ног.

Он был торжественный и нежный,
Когда впервой звучал для вас.
В науки океан безбрежный
Толпой входили в первый класс.

Прошли года. Труды, мученья
Остались грудой позади.
И все отчёты за ученье
У вас, конечно, впереди.

Чтоб аттестат вручили в руки,
Вам предстоит огромный труд:
Экзамен сдать за все науки,
А в них подъём довольно крут.

Преодолев барьер последний,
Вы беспорядочной толпой
Прибудете на бал последний ---
Традиционный, выпускной.

\date Май 1996 г.

\eject

\kern-1cm

\title ВЫПУСКНИКАМ

Не лебёдушки, не птицы
Белой стаей над водой,
В бальных платьицах девицы
Собрались на выпускной.

Ну и парни --- те, что надо,
Ходят, гордости полны.
Для родителей отрада ---
Лучше всех у них сыны.

Взяты школьные науки,
Все экзамены сданы.
В жизнь --- смелей, и счастье в руки,
Быть полезным для страны.

И отрадно, и тоскливо:
Всё придётся изменить.
Но привычной дружбы милой,
Школьной жизни не забыть.

И при встречах не забылось,
Что у каждого из вас
Учащённо сердце билось,
Приходило в резонанс.

Кем бы в жизни вы ни стали,
Кто скромнее, кто смелей,
Лишь бы нас не забывали,
Школу, класс, учителей.
\kern-7pt
\date 1996 г.

\eject
\kern-1.9cm
\def\leaderfill{\leaders\hbox to 1em{\hss.\hss}\hfill}
\def\conline{\hbox to10.5cm}
\font\confont=omssdc10 at 12pt
\hskip2cm {\bf Содержание}

\kern-19pt

\noindent {\confont Есть прелесть в жизни деревенской}

\conline{Зимой в деревне\leaderfill 4}%
\conline{За селом\leaderfill 5}%
\conline{Весенние закаты\leaderfill 6}%
\conline{На сенокосе\leaderfill 7}%
\conline{Тырбыл\leaderfill 8}%
\conline{На рыбалке\leaderfill 10}%
\conline{Ночёвка у рыбацкого костра\leaderfill 12}%
\conline{К 70-летию района\leaderfill 14}%
\conline{В родной деревне\leaderfill 16}%
\conline{Памяти Е. И. Мухортова\leaderfill 17}%
\conline{Кафе\leaderfill 18}%
\conline{Женщинам\leaderfill 19}%
\conline{Необходимый человек\leaderfill 20}%
\conline{После дождя\leaderfill 21}%
\conline{Бабушка и козы\leaderfill 22}%

\noindent {\confont То было трагедией века}

\conline{Солянский батальон\leaderfill 24}%
\conline{Памятник\leaderfill 25}%
\conline{Мой орден\leaderfill 26}%
\conline{На встрече ``Огненные дороги''\leaderfill 27}%
\conline{Другу, ветерану, учителю А. Д. Макаренко\leaderfill 29}%
\conline{Дедушка и внуки\leaderfill 30}%
\conline{Дети войны\leaderfill 31}%
\conline{Солдатская поэма\leaderfill 33}%
\conline{Родственнице погибшего солдата\leaderfill 35}%
\conline{Моё поколение\leaderfill 39}%

\eject

\kern-1.1pt

\noindent {\confont Школа --- моя судьба}

\conline{Новая школа\leaderfill 42}%
\conline{Школьникам о Пушкине\leaderfill 44}%
\conline{Вечер встречи\leaderfill 45}%
\conline{Моим ученикам\leaderfill 47}%
\conline{Старый тополь\leaderfill 48}%
\conline{Письмо бывшему ученику\leaderfill 49}%
\conline{Расставание\leaderfill 50}%
\conline{Таня\leaderfill 52}%
\conline{Оперились скворушки\leaderfill 53}%
\conline{Свадьба\leaderfill 54}%
\conline{День воспоминаний\leaderfill 55}%
\conline{Последний звонок\leaderfill 56}%
\conline{Выпускникам\leaderfill 57}%

\bye
