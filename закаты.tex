\input 12pt
\advance\pdfvorigin by -2cm
\advance\vsize by 3cm 
\def\pagebody{\vbox to\vsize{\boxmaxdepth=\maxdepth
  \borderone
  \nointerlineskip
  \bordertwo
  \nointerlineskip
  \moveright3cm \box255 \vss}}
\dimen0=1.7pt % width of outer border
\def\borderone{\vtop to0pt{
  \hrule height0pt
  \vbox{
    \moveright2\dimen4 \vbox{\advance\hsize by-4\dimen4 \hrule height\dimen0 width\hsize}
    \nointerlineskip
    \kern-\dimen0
    \nointerlineskip
    \moveright\dimen8 \vbox{\hrule height\dimen6 width\dimen0 }
    \nointerlineskip
    \kern-\dimen0
    \nointerlineskip
    \moveright\dimen8 \vbox{\hrule height\dimen0 width\dimen6 }
    \nointerlineskip
    \kern-\dimen6
    \nointerlineskip
    \moveright2\dimen4 \vbox{\hrule height\dimen6 width\dimen0 }
    \nointerlineskip
    \kern-\dimen0
    \hrule height\dimen0 width\dimen6
    \kern-\dimen0
    \line{\advance\vsize by-4\dimen4 \vrule height\vsize width\dimen0 \hfil\vrule width\dimen0 }
    \kern-\dimen0
    \hrule height\dimen0 width\dimen6
    \kern-\dimen0
    \nointerlineskip
    \moveright2\dimen4 \vbox{\hrule height\dimen6 width\dimen0 }
    \nointerlineskip
    \kern-\dimen6
    \nointerlineskip
    \moveright\dimen8 \vbox{\hrule height\dimen0 width\dimen6 }
    \nointerlineskip
    \kern-\dimen0
    \nointerlineskip
    \moveright\dimen8 \vbox{\hrule height\dimen6 width\dimen0 }
    \nointerlineskip
    \kern-\dimen0
    \nointerlineskip
    \moveright2\dimen4 \vbox{\advance\hsize by-4\dimen4 \hrule height\dimen0 width\hsize}
  }\vss}}
\dimen2=.8pt % width of inner border
\dimen4=5pt % distance of inner border from sides
\dimen6=2\dimen4 \advance\dimen6 by\dimen0 % for overlap
\dimen8=\hsize \advance\dimen8 by-2\dimen4 \advance\dimen8 by-\dimen0
\def\bordertwo{\vtop to0pt{
  \hrule height0pt
  \kern\dimen4
  \moveright\dimen4 \vbox{\advance\hsize by-2\dimen4
    \hrule height\dimen2
    \kern-\dimen2
    \line{\advance\vsize by-2\dimen4 \vrule height\vsize width\dimen2 \hfil\vrule width\dimen2 }
    \kern-\dimen2
    \hrule height\dimen2
  }\vss}}
{\catcode`\^^M=\active
\gdef\title#1^^M{{\bf#1}\par}
\gdef\date#1^^M{\hbox to10cm{\hss\it#1}\par}}
\def\par{\leavevmode\endgraf} \obeylines \let\par=\endgraf

\kern1cm

\title ЗИМОЙ В ДЕРЕВНЕ

У ёлок варежки из ваты,
Берёзки мёрзнут на юру.
Ворон, синичек мир крылатый
Поближе тянется к двору.

Который день пурга гуляет,
Стучит крупинками в окно,
Живую влагу обращает
В полупрозрачное стекло.

Мороз снаружи. Зябко в доме.
Нарушен жизненный уют.
Где подевались те циклоны,
Которые тепло несут?

Приходит ночь со снежным валом,
Стучит, гудит в печной трубе.
И мёрзну я под одеялом
В моей остуженной избе.

\date Февраль 1988 г.

\vfill\eject

\title ЗА СЕЛОМ

За селом Солянкой ---
Колки да полянки.
Сколько глаз хватает ---
Тучные поля.
Жаворонки в небе,
Вольный ветер в хлебе.
Здесь и начинается
Родина моя.

Сразу на опушке
Слышен зов кукушки.
В ельнике меж кочек
Ручеёк журчит.
Запахи цветочные,
Звуки полуночные.
На меня вселенная
Звёздами глядит.

Лай собак в деревне,
В ветках пташек пенье.
Караси играют
В утреннем пруду.
Солнце и прохлада ---
Всё, что сердцу надо.
Не зовите в город,
В город не пойду.

Сразу за Солянкой ---
Колки да полянки.
От цветов и красок
Аж в глазах рябит.
Край ты наш привольный,
Мы тобой довольны.
Нам тобой гордиться,
Нам тебя любить.

\date Январь 1985 г.

\vfill\eject

\title ВЕСЕННИЕ ЗАКАТЫ

Какие дивные закаты
У нас весною за селом!
А солнце кажется девчатам,
Как медь, натёртая песком.

Играют, бегают ребята.
И с ними тени их снуют.
Косые тени-великаны
Такую же игру ведут.

В лесу, где царство Берендея,
В цветах, где травка на лугу,
Под взглядом глаз твоих робею,
Но, что прикажешь, всё смогу.

Не каркай ты, треклятый ворон,
И пусть закроет хищник пасть.
Я за тебя отваги полон,
И волоску не дам упасть.

Какие дивные закаты.
Дрожат последние лучи.
Как в сказку, мысль несёт куда-то,
И я молчу\dots И ты молчи\dots

\date 1985 г.

\vfill\eject

\title НА СЕНОКОСЕ

Хорошо на сенокосе!
Утомлённый, чуть живой,
Я бреду в шалаш к откосу
С родниковою водой.

День ушёл с рабочей песней,
Ночь крадётся на луга.
На погоду в поднебесье
Месяц выставил рога.

Чувства гордость растоптали.
Я гляжусь в твоё лицо.
Мне кузнечики сковали
Обручальное кольцо.

Я на всё с тобой согласен,
Коль душа лежит к душе.
С милой белый свет прекрасен,
С нею рай и в шалаше.

И усну опустошённый
Я со звоном комара.
Очарованный, влюблённый,
Чтобы сил набраться до утра.

\date 1986 г.

\vfill\eject

\title ТЫРБЫЛ

За деревней Спасовкой,
На ручье Тырбыл
Не однажды в праздники
Я с друзьями был.

Струи серебристые
Бьют из-под горы.
В тихих водах чистые
Видятся миры.

По весне черёмуха
Воздухом пьянит,
А вода без отдыха
По камням журчит.

По полям крутятся
Тучные стада,
В водопоях мутится
Чистая вода.

В заводях без камешков
Травы шелестят,
Уточка-хозяюшка
Сторожит утят.

Ниже по течению,
Где была Ловать,
Красную смородину
Едут люди брать.

Там поля клубничные,
По лесам грибы.
Берега приличные
Выставили лбы.

Где теченье быстрое,
Берега узки,
Там живые молнии ---
Рыбы харьюзки.

\vfill\eject

За деревней Спасовкой,
На ручье Тырбыл,
Отдыхая в праздники,
Я уху варил.

Хочешь видеть чудо,
Поднебесный рай?
Приезжай! Иуда,
Но не предавай.

\date 1988 г.

\vfill\eject

\title НА РЫБАЛКЕ

Когда закончилась неделя,
И вы хотите отдохнуть,
Готовьте удочку немедля,
Спешите на гмирянский пруд.

И здесь, и там стоят машины.
Безвучно дремлют тростники.
Вдали у зарослей кувшинок
Сидят на лодках рыбаки.

Всплеснёт карась. И тихо снова,
Лишь по воде круги пойдут,
Но от события такого
У рыбаков сердца замрут.

Вот поплавок опять качнулся,
Подсечка! Удочка --- в дугу\dots
Сосновый бор уже проснулся,
На том чернеет берегу.

Лишь первый луч осветит воду,
В костре погаснет уголёк,
По чутким клавишам природы
Промчится сонный ветерок.

Плывет рыбак --- домой собрался.
И сразу разговор пошел:
--- Досадно, вот такой! Сорвался\dots
--- И у меня большой сошёл\dots

Вот осмотрел рыбак машину,
Сложил все снасти: стой - не стой\dots
Прошёл, ногой потрогал шину,
Завёл --- и тронулся домой.

Над лесом каркает ворона,
Повисли звуки над селом.
Сосна отряхивает крону,
И радость шепчется с добром.

\vfill\eject

\kern-1.3cm

\title НОЧЁВКА У РЫБАЦКОГО КОСТРА

Хлеба нарезали всем по горбушке,
Плащ расстелив у костра на траве.
Водку распили из старенькой кружки.
Кровь разыгралась. Туман в голове.
Все нахлебались горячей ушицы.
В теле истома и липкая лень.
Лишь балагуры плетут небылицы,
Да в голове над сознанием тень.
Где-то кричит коростель на болоте.
Мирно гудят комары.
Ждите хоть год, никогда не найдете,
Лучше, чем эта, поры.
Гаснет костёр. Только искры метутся.
Хочется спать\dots{\spacefactor=3000} Да какой там уют?!
Дивы ночные за спиной крадутся,
Схватят, с собой унесут.
Сон одолел. В разных позах уснули.
Пала роса на траву.
Ночью озябли ребята. Проснулись,
Тянутся ближе к костру.
Одеяло утро сбросило.
Темнота раскрыла даль.
Небеса закупоросило.
В небытье ушла печаль.
Расплескался свет по воле.
Счастьем вспыхнула душа.
Летом русская природа
До чего же хороша!
Птичьи трели разливаются.
И в реке вода журчит.
Вальсом воздух наполняется,
Вороньё в лесу кричит.
Солнце выше поднимается.
День рабочий настаёт.
Кто-то с кем-то разлучается,
А кого-то кто-то ждёт.
Снасти сложили в машину.
Сели, тронулись домой,
А дорогой порешили\dots
Снова ехать в выходной.

\vfill\eject

\title К 70-ЛЕТИЮ РАЙОНА

Между Канской землёй и Уярской
Расположился наш район.
По законам властей пролетарских
Назван Рыбинским и утверждён.

На Московской дороге бескрайней,
От железных дорог вдалеке,
Село Рыбное стало центральным,
Что стояло на Рыбной реке.

Кто пешком, кто богаче --- на дрогах,
Прихвативши с собой ``сидорки'',
По разбитым сибирским дорогам,
Добирались туда мужики.

``До чего неудобственно, паря, ---
Говорил мужик мужику, ---
Время нет, а неделями тратишь
По малейшему пустяку.

Вот железной дорогой удобнее
И быстрей, и короче пути.
Есть ведь Троицко-Заозёрное,
Центр бы туда перенести.''

Так и сделали умные люди.
Сразу жизнь веселее пошла.
Поездами добраться нетрудно
Из любого глухого села.

Заозёрный --- большая деревня.
Как во всякой сибирской глуши,
По пригоркам стояли деревья,
По болотам росли камыши.

Годы шли, и леса вырубались.
Отходила всё дальше тайга.
Хорошело село, разрасталось,
Истощалась, мелела Барга.

\vfill\eject

Изобильный район и богатый,
Очень важный для жизни страны.
Люди в нем работящие, знатные,
Как в труде, так на поле войны.

Добывается уголь в разрезах.
Созревают хлеба на полях.
На лугах стада тёлочек резвых,
И в достатке зерно в закромах.

Заозёрный теперь уже город ---
В сорок пять исторических лет.
Среди прочих российских он молод
И трамвая пока ещё нет.

Но я верю, со временем будет,
Что для города быть суждено.
В нём живут настоящие люди,
И с прогрессом они заодно.

\vfill\eject

\title В РОДНОЙ ДЕРЕВНЕ

Есть прелесть в жизни деревенской,
Свои особые черты
В людской любви и ласке женской.
Я здесь с природою на ``ты''.

Нависли ивы над водою.
Паук для мухи сети вьет.
Связал два облака чертою
Летящий в небе самолёт.

Гусак шипит, гогочет стая,
Петух подрался и поёт.
Из-под ворот собака злая
Кому-то ``интервью'' даёт.

Ликует жизнь, исчезла злоба.
Мать ребятне наказ даёт:
``Играйте, да смотрите в оба ---
Чёрт за углом проступка ждёт''.

Когда тоска одолевает,
Пойду по улице пройду\dots
Я знаю всех, меня все знают ---
Здесь каждый житель на виду.

Ходил, летал, бывал на море ---
Таков уж наш двадцатый век.
И только здесь я что-то стою,
Лишь здесь я --- нужный человек.

\date 1992 г.

\vfill\eject

\title ПАМЯТИ Е. И. МУХОРТОВА

Давайте вспомним и о нём.
На стол блинов поставим горку
И добрым словом помянём,
Как жил наш друг Е.И.Мухортов.
Он был директором ДК,
Работою своей гордился.
Дни напролёт и вечера
Среди народа находился.
У женщин слыл за чудака:
До дней последних не женился.
Среди мужчин --- за простака:
Бесплатно, ревностно трудился.
Весной с лопатою ходил,
Рыл землю и готовил ямки.
Потом деревья в них садил
На улицах родной Солянки.
Минули годы, лес растёт.
Весною птицы прилетают.
Теперь Мухортова уж нет,
Но сделанное им не умирает.
Пройдись по улицам села:
Одна другой светлей и краше.
Смолистый дух даёт сосна,
Косынками берёзки машут.
И кто бы что ни говорил,
Народам ведано издревле:
``Тот не напрасно век прожил,
Кто для людей сажал деревья''.
Мы перед памятью в долгу,
Портрет в ДК повесим в рамке
И надпись сделаем к нему:
``Почётный гражданин Солянки''.

\date 1988 г.

\vfill\eject

\title КАФЕ

По селу плывут слухи разные
И несут они вести важные.
Сватье сват брюзжит, распинается:
--- Молодёжь у нас дурью мается.

Пусть потешатся злопыхатели,
А что скажут нам доброжелатели?
--- Молодежь у нас не ленивая,
Беспокойная и счастливая.

Что ни юноша --- первым кажется.
Что ни девица --- то красавица.
Сам комсорг-вожак с комсомольцами
Подвизались быть добровольцами.

Они сделали всё возможное,
Чтоб открыть кафе молодёжное.
Дни до полночи мыли, красили
И в труде своем были счастливы.

Заходи, народ, волей вольною
В наш салон-кафе безалкогольное.
Распивай чаи, кушай пряники,
Хмурый зимний день станет праздником.

\vfill\eject

\title ЖЕНЩИНАМ

Весну я встречаю, как счастье,
В неё, видно, все влюблены.
Конечно, бывают ненастья,
Но чаще погожие дни.
Белее и чище берёзы,
Заметно прибавились дни.
Ещё не ослабли морозы,
Но видится поступь весны.
У крыш кружевные сосульки,
Шустрее снуют воробьи.
А утром хрустят в переулке
Поспешные чьи-то шаги.
И бабушки наши, и мамы,
Подруги, что с нами живут,
Влюблёнными смотрят глазами,
Подарков и праздника ждут.
Я вам по секрету открою:
Товар на прилавках лежит,
Но хочется чего-то такого,
Чтоб радостно было дарить.
Все добрые женщины --- милы.
Чужие, свои --- всё равно.
Жалею когда-то счастливых,
Покинутых кем-то давно.
Пусть радостно светятся лица,
Серебряный сыплется смех.
Пусть горе уйдёт небылицей,
И счастье прибудет для всех.
Целую вас нежно и жарко ---
Друзья не сочтут за грехи.
Пусть будут скромным подарком
Душевные эти стихи.

\date Март 1986 г.

\vfill\eject

\title НЕОБХОДИМЫЙ ЧЕЛОВЕК

Морозным днём иль в летний зной,
Где рельсы тянутся по шпалам,
Стучит ключом или киркой
Рабочий Ваня Шаповалов.
Монтёр, как все, работу знает,
Но есть особые штрихи:
Завидным даром обладает
Из слов монтировать стихи.
Куда он думы устремляет?
Куда ведёт его звезда?
Холодным ветром обдавая,
Проходят с шумом поезда\dots
На уши давит близкий грохот,
И рельсы гнутся и дрожат.
Из-под колёс, рифмуя строки,
К нему метафоры летят.
В словесном вихре мысль порхает,
Он ловит трепетной рукой.
И никого не замечая,
Строфу черкает за строфой.
Светлей становится повсюду,
И радостней тяжёлый труд.
Прекрасно, что такие люди
Здесь с нами рядышком живут.
В тяжёлом грохоте металла
Кончается двадцатый век.
И в этой жизни Шаповалов ---
Необходимый человек.

\vfill\eject

\title ПОСЛЕ ДОЖДЯ

Дождик кончился. Тепло.
На дорожках лужицы.
А водичка в ручейках
Пенится и кружится.
Мы --- ребята-моряки,
Не боимся мы реки.
Мы --- ребята смелые,
По водичке бегаем.
Мамочка стучит в окно,
Бабушка ругается ---
Это внучка у неё
Грязнулей называется.
Только папа не бранит,
Папа улыбается:
И девчонке смелой быть
Тоже полагается.
Мне нисколечко не больно,
Я упала только раз.
Если б были все довольны,
Я б ещё похлюпалась.

\vfill\eject

\title БАБУШКА И КОЗЫ

Козочки у бабушки
Весело живут.
Щиплют вволю травушку,
Молоко дают.
Мы с братишкой рядышком
К бабушке пойдем.
Поедим оладушек,
Молочка попьём.
На лужайке с бабушкой
Козочек пасём,
Посидим с ней рядышком,
Песенки споём.
Любим сказки страшные ---
Их не перечесть.
Хорошо, что бабушки
В нашей жизни есть.

\vfill\eject

\title СОЛЯНСКИЙ БАТАЛЬОН

Их было три сотни, почти батальон,
Солдат из деревни родной.
На них похоронки разнес почтальон.
Тела их в могилах давно.

То были все парни и мужики.
Деревня без них опустела.
Остались солдатки и старики,
Да малые дети в постелях.

Под Ржевом зимою в пургу и мороз
Сибирские Гришки и Ваньки,
3акаменев от обиды и слёз,
Шли с кулаками на танки.

Тогда их немного осталось в живых:
Давили немецкие танки.
Всю мощь огневую нацелили в них---
Солдат из деревни Солянка.

Нет-нет, да и вспомнят родные о них.
Висят фотографии в рамках,
Да старые письма о днях фронтовых
Мужчин, так любивших Солянку.

Давайте наполним стаканы вином
В день нашей великой Победы,
Помянем погибших и вспомним о том,
Как прожили мы эти беды.

Мы жизнью обязаны тем, кто погиб.
Пусть памятник напоминает
О тех, кто в далеких могилах лежит.
Их помнят, а, вспомнив, рыдают.

\vfill\eject

\title ПАМЯТНИК

Пускай здесь не было боёв,
Не прорывались вражьи танки.
В одном из лучших уголков
Воздвигнут памятник в Солянке.

За плечи вскинут автомат,
И с обнажённой головою
Стоит, задумавшись, солдат---
Он только вышел с поля боя.

Никто не скажет точно, чей он.
А старожилы говорят:
``Один из Калиниченко
Или из Котовых ребят\dots''

И кто такой на пьедестале?
Уж мало тех, кто помнит их.
``У этих-то двоих не стало\dots
А вот у Котовых --- троих\dots''

И всё равно не обознались.
Пусть будет этот или тот.
Всего и Солянку не вернулось
Примерно около трёхсот.

Молчит солдат, и всё смолкает.
Всплывают память и мечты.
Молодожёны тут бывают,
И дарят девушки цветы.

Пусть он недвижен, пусть молчит.
Солдат, сержант --- суть не в званьи,
Погибшим в память он стоит,
А всем живущим --- в назиданье.

А где-то и сейчас бои
Снаряды рвутся, мчатся танки,
И памятник у нас стоит
Возле ДК среди Солянки.

\vfill\eject

\title МОЙ ОРДЕН

Мне орден лучистый вручила страна.
В нём надпись словами по кругу:
``ОТЕЧЕСТВЕННАЯ ВОЙНА''.
Я был среди тех, кто за Родину встал,
Когда враг жестокий коварно напал.
Коробился, гнулся металл.
Воронки, окопы, блиндажи\dots
В разрывах снарядов, под танком лежи,
В атаку вставай, не дрожи.
Отцы наши там погибли, друзья.
Пусть будет им пухом родная земля.
И чудом остался лишь я.
И вот за себя и за них я живу,
Работаю честно и мир сторожу.
И орден по праву ношу.

\vfill\eject

\title НА ВСТРЕЧЕ ``ОГНЕННЫЕ ДОРОГИ''

В деревне с рассветом поют петухи,
Заботы ложатся на плечи.
Я маюсь всю ночь, сочиняю стихи,
Пишу о торжественной встрече.

Девчонки и парни --- студенты они,
А это великая силища.
Решили собрать ветеранов войны ---
Выпускников педучилища.

Когда над страной разразилась война,
Никто не остался в сторонке.
Всех взрослых мужчин подбирала она
И слала взамен похоронки.

3а Каном-рекою, в густом сосняке
Отцы становились солдатами.
И канули в вечность, как слёзы в реке,
Которыми матери плакали.

На встрече сидели одни старики
И юность свою вспоминали,
Как в тяжкие годы военной поры
И мёрзли, и голодали.

Несчётное множество день ото дня
Критических было моментов.
Надёжною выручкой было тогда
Великое братство студентов.

Солидные дамы --- девчонки в те дни
В фуфайках одетые спали.
В нетопленых классах учились они,
На книжках, газетах писали.

А эта!\dots{\spacefactor=3000} Я к ней подступиться не смел:
Глядела надменно и косо
На парня, что сзади за партой сидел,
И трогал за длинные косы.

\vfill\eject

Ребята, что были на фронте тогда,
Такие, как внуки сегодня,
Теперь --- ветераны войны и труда,
Высоких наград удостоены.

В той группе смеются, а в этой грустят,
Платками глаза вытирают.
Друзей и любовь не воротишь назад,
А что впереди ожидает?\dots

Уже наступает предутренний час,
А мы даже глаз не сомкнули.
Спасибо, ребята, за то, что вы нас
Опять в нашу юность вернули.

Пусть жизнь ваша будет
Без трещин и ран.
Учитесь, трудитесь без лени.
Я слышал, когда говорил ветеран:
``Для нас вы --- достойная смена''.

\date Март 1985 г.

\vfill\eject

\title ДРУГУ, ВЕТЕРАНУ, УЧИТЕЛЮ \endgraf %
\ \ \ \ \ \ \ \ \ А.Д. МАКАРЕНКО

И до войны ты был учитель,
Лишь старше стал, да больше стаж.
Наш воин --- скромный победитель,
А для Отечества --- солдат.

Ты в праздник надеваешь китель.
До блеска чистишь ордена.
И порассказывать любитель
За стопкой крепкого вина.

Картечь спасала батарею.
Враги трезвели, шли назад.
И грозен был в дыму сражений
Пехоты нашей резкий мат.

Не покидая батарею,
И при разрывах вражьих мин,
Освобождал Варшаву, Вену,
А в мае штурмом брал Берлин.

Судьба тебя не баловала,
А всё-таки тебе везло.
Из трёх полученных ранений,
Два оказалися легко.

Всё время подвергаясь смерти,
Почти не спал, не часто ел.
Таких, как ты, не брали черти,
Ты выжил, ты остался цел.

Учитель, как и был учитель\dots
Судьбой доволен, жизни рад,
Наш воин --- скромный победитель,
А для Отечества --- солдат.

\date 1988 г.

\vfill\eject

\kern1.2cm

\title ДЕДУШКА И ВНУКИ

Из садика вечером внуки пришли
И деда упрашивать стали:
``Ты нам о войне, о себе расскажи,
За что ордена и медали?''

А дедушка внуков руками обнял,
Стал грустным, подумал немного,
Вздохнул глубоко и тихонько начал:
``Я вам не желаю такого.

Когда над страной разразилась война,
Мы в старшие классы ходили.
Тяжёлые были тогда времена,
И все в напряжении жили.

От взрывов и крови стонала земля
От Белого моря до Чёрного.
Сражались солдаты, держалась Москва,
И делали всё невозможное.

А матери наши в полях, у станков,
Работали и голодали,
Мужей заменив, а мальчишки --- отцов,
Для фронта победу ковали.

Не год и не два, а четыре прошло,
И мы в своё время успели
В атаки сходить. Только нам повезло:
Остались ранения в теле.

\vfill\eject

\kern.6cm

Пятьдесят миллионов людей
В войне этой в землю упрятаны.
Неужто найдётся, как Гитлер, злодей,
Который мечтает об атомной? \dots

Идите, ребятки, а я посижу.
Ложитесь и спите спокойно.
А я, если надо, ещё послужу,
Чтоб сгинули всякие войны.''

\date Февраль 1986 г.

\title ДЕТИ ВОЙНЫ

Не мерить кровь и слезы тоннами\dots
Кому и сколь пришлось страдать.
И на машине электронной
Статистикам не сосчитать.

Мы шли на фронт землёй Смоленщины,
Минуя минные поля,
Участливо глядели женщины
В освобождённых деревнях.

То не деревни --- лишь названия,
Испепелённая земля\dots
Стояли печки --- изваяния
В местах, где улица была.

И нам, недавним малолеткам,
Был за отцами путь один\dots
Но мы шутили не по-детски:
``Где тут дорога на Берлин?''

\bye
