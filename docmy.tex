%TODO: use code from here directy in documents - delete this file

\def\today{\number\year-\twodigits=\month \printtwodigits-\twodigits=\day \printtwodigits}
\newcount\twodigits
\def\hours{\twodigits=\time \divide\twodigits by 60 \printtwodigits
  \multiply\twodigits by-60 \advance\twodigits by\time :\printtwodigits}
\def\gobbleone1{}
\def\printtwodigits{\advance\twodigits100
  \expandafter\gobbleone\number\twodigits
  \advance\twodigits-100 }

\newcount\ttlineno
\def\Blue{} % TODO: purge this everywhere - it is stub for dvi mode
\def\Red{} % TODO: purge this everywhere - it is stub for dvi mode
\def\Brown{} % TODO: purge this everywhere - it is stub for dvi mode
\def\Green{} % TODO: purge this everywhere - it is stub for dvi mode
\def\Yellow{} % TODO: purge this everywhere - it is stub for dvi mode
\def\Black{} % TODO: purge this everywhere - it is stub for dvi mode
\let\oriBlack=\Black % TODO: purge this everywhere - it is stub for dvi


          \def\titmodule{Modul}
          \def\tittoc{Obsah}
          \def\titindex{Rejstrik}
          \def\opartname{>> CAST}

\def\titversion{}
\def\projectversion{\today\ \hours}


%%%%%%%%%%%%%%%%%% Parametry a pomocná makra pro nastavení vzhledu

\parindent=30pt

\newdimen\nwidth \nwidth=\hsize \advance\nwidth by-2\parindent

\raggedbottom
\exhyphenpenalty=10000

\font\bbf=omb10 at12pt
\font\bbbf=omb10 at14.4pt
\font\btt=omtt12
\font\ttsmall=omtt8
\font\rmsmall=omr8
\font\itsmall=omti8
\font\partfont=omb10 at80pt

\def\setsmallprinting{\ttsmall \let\it=\itsmall \let\rm=\rmsmall 
   \def\ttstrut{\vrule height8pt depth3pt width0pt}%
   \offinterlineskip \parskip=-1pt\relax
}
\def\setnormalprinting{\tt \baselineskip=0pt \hfuzz=4em 
   \def\ttstrut{\vrule height10pt depth3pt width0pt}%
   \offinterlineskip \parskip=-1pt\relax
}

   \def\rectangle#1#2#3#4{\vbox to0pt{\vss\hrule\kern-.3pt
      \hbox to#3{\vrule height#1 depth#2\hss#4\hss\vrule}%
      \kern-.3pt\hrule\kern-#2\kern-.1pt}}

\def\docbytex {\leavevmode\hbox{DocBy}.\TeX}

%%%%%%%%%%%%%%%%%% Vzhled kapitol a podkapitol

\def\printsec #1{\par
    \removelastskip\bigskip\medskip
    \noindent
    \rectangle{16pt}{9pt}{25pt}{\Brown\bbbf\ifsavetoc\the\secnum\else\emptynumber\fi}%
    \kern5pt{\bbbf #1}\par
}
\def\printsecbelow {\nobreak\medskip}

\def\printsubsec #1{\par
    \removelastskip\bigskip
    \noindent
    \vbox to0pt{\vss
       \rectangle{16pt}{9pt}{25pt}{\Brown\bf
          \ifsavetoc\the\secnum.\the\subsecnum\else\emptynumber\fi}\kern-5pt}%
    \kern5pt{\bbf \let\tt=\btt #1}\par
}
\def\printsubsecbelow {\nobreak\smallskip}

\def\printpart #1{\par
    \removelastskip\bigskip\medskip
    \noindent {\linkskip=60pt }%
    \rectangle{16pt}{9pt}{25pt}{}%
    \kern-20pt{\Brown\partfont\thepart\Black}\kern10pt{\bbbf #1}\par
}
\def\printpartbelow {\nobreak\bigskip}

\def\emptynumber{}

%%%%%%%%%%%%%%%%%%%% Titul, autor

\def\title{\def\tmpA{title}\futurelet\nextchar\secparam}
   \def\iititle {\par
      \ifx\projectversion\empty \else 
        \line{\hss\rmsmall\titversion\projectversion}\nobreak\medskip\fi
      \centerline{\bbbf \the\sectitle}\nobreak\medskip}

\ifx\projectversion\undefined \def\projectversion{}\fi

%%%%%%%%%%%%%%%%%%%% Footline

\footline={\hss\rectangle{8pt}{2pt}{25pt}{\tenrm\Black\folio}\hss}

%%%%%%%%%%%%%%%%%%%% Tisk z prostředí \begtt, \endtt

   \def\printvabove{\smallskip\hrule\nobreak\smallskip\setnormalprinting}
   \def\printvbelow{\nobreak\smallskip\hrule\smallskip}
   \def\printvline#1#2{\noindent
     \ifx\islinenumprinted+
       \hbox to13pt{\hss#1:\kern4pt}%
     \else
       \kern17pt
     \fi
     {\ttstrut#2\par}\penalty12 }

%%%%%%%%%%%%%%%%%%%%%%% Pomocna makra

\def\defsec#1{\expandafter\def\csname#1\endcsname}
\def\edefsec#1{\expandafter\edef\csname#1\endcsname}
\def\undef#1\iftrue{\expandafter\ifx\csname#1\endcsname\relax}

{\catcode`\^^I=\active \gdef^^I{\space\space\space\space\space\space\space\space}
 \catcode`\|=0 \catcode`\\=12 |gdef|nb{\}}
\bgroup
   \catcode`\[=1 \catcode`]=2 \catcode`\{=12 \catcode`\}=12 \catcode`\%=12
   \gdef\obrace[{] \gdef\cbrace[}] \gdef\percent[%]
\egroup
\def\inchquote{"}

\def\uncatcodespecials{\def\do##1{\catcode`##1=12}\dospecials}

%%%%%%%%%%%%%%%%%% Příkazy \begtt, \endtt

\def\begtt {\bgroup\printvabove
   \uncatcodespecials \catcode`\"=12 \catcode`\^^M=12 \obeyspaces
   \setsmallprinting \startverb}
\def\begintt {\bgroup\printvabove
  \ttlineno=0
  \let\islinenumprinted+
  \uncatcodespecials \catcode`\"=12 \catcode`\^^M=12 \obeyspaces
  \setsmallprinting \startverb}
{\catcode`\|=0 \catcode`\^^M=12 \catcode`\\=12 %
  |gdef|startverb^^M#1\endtt{|runttloop#1|end^^M}%
  |gdef|runttloop#1^^M{|ifx|end#1 |expandafter|endttloop%
     |else|global|advance|ttlineno by1 %
       |printvline{|the|ttlineno}{#1}|relax|expandafter|runttloop|fi}} %
\def\endttloop#1{\printvbelow\egroup\futurelet\nextchar\scannexttoken}
\long\def\scannexttoken{\ifx\nextchar\par\else\noindent\fi}

\output={\outputhook \plainoutput }

%%%%%%%%%%%%%%%%%% Kapitoly, podkapitoly

\newcount\secnum
\newcount\subsecnum
\newtoks\sectitle
\newif\ifsavetoc \savetoctrue

\def\sec{\def\tmpA{sec}\futurelet\nextchar\secparam}  
\def\subsec{\def\tmpA{subsec}\futurelet\nextchar\secparam}

\def\secparam{\ifx\nextchar[%
     \def\tmp[##1]{\def\seclabel{##1}\futurelet\nextchar\secparamA}%
     \expandafter\tmp
   \else \def\seclabel{}\expandafter\secparamB\fi
}
\def\secparamA{\expandafter\ifx\space\nextchar 
      \def\tmp{\afterassignment\secparamB\let\next= }\expandafter\tmp
   \else \expandafter\secparamB \fi
}
\def\secparamB #1\par{\nolastspace #1^^X ^^X\end}
\def\nolastspace #1 ^^X#2\end{\ifx^^X#2^^X\secparamC #1\else \secparamC #1^^X\fi}
\def\secparamC #1^^X{\sectitle={#1}\csname ii\tmpA\endcsname}

\def\iisec{%
    \ifsavetoc \global\advance\secnum by1 \global\subsecnum=0 \fi
    \expandafter \printsec \expandafter{\the\sectitle}% vlozi horni mezeru, text, nakonec \par
    \savetoctrue \printsecbelow
}
\def\iisubsec {%
    \ifsavetoc \global\advance\subsecnum by1 \fi
    \expandafter \printsubsec \expandafter{\the\sectitle}% vlozi horni mezeru, text, nakonec \par
    \savetoctrue \printsubsecbelow
}

%%%%%%%%%%%%%%%%%%%%%%% Part (doplneno leden 2009)

\newcount\partnum
\def\thepart{\ifcase\partnum --\or A\or B\or C\or D\or E\or F\or G\or
  H\or I\or J\or K\or L\or M\or N\or O\or P\or Q\or R\or S\or T\or
  U\or V\or W\or X\or Y\or Z\else +\the\partnum\fi}

\def\part{\def\tmpA{part}\futurelet\nextchar\secparam}
\def\iipart{% 
    \ifsavetoc \global\advance\partnum by1 \fi
    \expandafter \printpart \expandafter{\the\sectitle}% vlozi horni mezeru, text, nakonec \par
    \savetoctrue \printpartbelow
}

%%%%%%%%%%%%%%%%%%%%%%% Odkazy, reference

\ifx\pdfoutput\undefined
   \def\savelink[#1]{}
   \def\ilink [#1]#2{#2}
   \def\savepglink{}
   \def\pglink{\afterassignment\dopglink\tempnum=}
   \def\dopglink{\the\tempnum}
   \def\ulink[#1]#2{#2}
\else
   \def\savelink[#1]{\ifvmode\nointerlineskip\fi
      \vbox to0pt{\def\nb{/_}\vss\pdfdest name{#1} xyz\vskip\linkskip}}
   \def\ilink [#1]#2{{\def\nb{/_}\pdfstartlink height 9pt depth 3pt
      attr{/Border[0 0 0]} goto name{#1}}\Blue#2\Black\pdfendlink}
   \def\savepglink{\ifnum\pageno=1 \pdfdest name{sec::start} xyz\relax\fi % viz \bookmarks
       \pdfdest num\pageno fitv\relax}
   \def\pglink{\afterassignment\dopglink\tempnum=}
   \def\dopglink{\pdfstartlink height 9pt depth 3pt
       attr{/Border[0 0 0]} goto num\tempnum\relax\Blue\the\tempnum\Black\pdfendlink}
   \def\ulink[#1]#2{\pdfstartlink height 9pt depth 3pt 
   user{/Border[0 0 0]/Subtype/Link/A << /Type/Action/S/URI/URI(#1)>>}\relax
   \Green{\tt #2}\Black\pdfendlink}
\fi
\newdimen\linkskip \linkskip=12pt

\def\reflabel #1#2#3{%
   \undef{^^Y#1}\iftrue
      \ifx^^X#2^^X\else\defsec{^^X#1}{#2}\fi
      \defsec{^^Y#1}{#3}%
   \else
      \dbtwarning{The label [#1] is declared twice}%
   \fi
}
\def\numref [#1]{\undef{^^X#1}\iftrue \else \csname^^X#1\endcsname\fi}
\def\pgref [#1]{\undef{^^Y#1}\iftrue-1000\else \csname^^Y#1\endcsname\fi}

\def\labeltext[#1]#2{\savelink[#1]\writelabel[#1]{#2}}
\def\writelabel[#1]#2{\edef\tmp{\noexpand\writelabelinternal{#2}}\tmp{#1}}
\def\writelabelinternal#1#2{\write\reffile{\string\reflabel{#2}{#1}{\the\pageno}}}

\def\label[#1]{\labeltext[#1]{}}

\def\cite[#1]{\ifnum \pgref[#1]=-1000
      \dbtwarning{label [#1] is undeclared}\Red??\Black
   \else \edef\tmp{\numref[#1]}%
      \ifx\tmp\empty \edef\tmp{\pgref[#1]}\fi
      \ilink[#1]{\tmp}%
   \fi
}

\def\api #1{\label[+#1]\write\reffile{\string\refapiword{#1}}}
\def\apitext{$\succ$}

%% final settings

\catcode`\"=\active
\def"{\leavevmode\hbox\bgroup\let"=\egroup 
      \def\par{\errmessage{\string\par\space inside \string"...\string"}}%
      \uncatcodespecials\tt}

\def\langleactive{\uccode`\~=`\<\catcode`\<=13
      \uppercase{\def~}##1>{{$\langle$\it##1\/$\rangle$}}}

\font\eightrm=omr8
\font\eighttt=omtt8
\font\eightbf=ombx8
\font\eightsl=omsl8
\font\eightit=omti8

%FIXME: why it stops working if we put \baselineskip and \eightrm inside
%\begingroup-\endgroup to avoid using \normalbaselines\rm?
%(for example see \eighttt in tex/listing.tex - why it works?)
\def\begincomment{\baselineskip9pt\eightrm\begingroup
  \def\rm{\eightrm}%
  \def\tt{\eighttt}%
  \def\sl{\eightsl}%
  \def\bf{\eightbf}%
  \def\it{\eightit}%
  \parindent=0pt}
\def\endcomment{\endgroup\vskip1mm\normalbaselines\rm}

\def\setupverbatim{\setsmallprinting
  \def\par{\leavevmode\endgraf}
  \catcode`\`=\active
  \catcode`\"=12 % unactivate
  \obeylines
  \uncatcodespecials
  \obeyspaces
  \everypar{\ttstrut}}
{\obeyspaces\global\let =\ } % let active space = control space
{\catcode`\`=\active \gdef`{\relax\lq}}
\def\listing#1{\begingroup\setupverbatim\input#1 \endgroup}
